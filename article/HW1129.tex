\documentclass{article}

\usepackage[utf8]{inputenc}
\usepackage{amsmath}
\usepackage{amsfonts}
\usepackage{amssymb}
\usepackage{graphicx}

\title{\vspace*{-3.5cm}Homework for Linear Algebra \\November 29, 2024} 
\author{Chengyu Zhang}
\date{}

\begin{document}

\maketitle

\textbf{Exercise 1.}\\
\begin{itemize}
    \item Assume the result of left side $[\mathbf{u}_1,\cdots , \mathbf{u}_m]$\[
    \mathbf{u}_i=\sum_{k=1}^{n}(A+A')(k,i)\mathbf{v}_i=\sum_{k=1}^{n}A(k,i)\mathbf{v}_i+\sum_{k=1}^{n}A'(k,i)\mathbf{v}{i}
    \]
    So the equation holds.\\
    \item Let $C=AB$, $C_{ij}=\sum_{q=1}^{m}A(i,q)B(q,j)$. Assume the result of left side $[\mathbf{u}_1,\cdots , \mathbf{u}_l]$.
    \[
    \mathbf{u}_i=\sum_{k=1}^{n}\sum_{q=1}^{m}A(k,q)B(q,i)\mathbf{v}_k
    \]
    Consider the right side \[
    \mathbf{u}'_i=\sum_{q=1}^{m}(\sum_{k=1}^{n}A(k,i)\mathbf{v}_k)B(q,i)=\mathbf{u}_i
    \]
    So the equation holds.
\end{itemize}

\textbf{Exercise 2.}\\
(i)We have $(x,y)=c_1\mathbf{v}_1+c_2\mathbf{v}_2$
\[
\left\{
    \begin{array}{ll}
        x=c_1+2c_2\\
        y=c_1+3c_2
    \end{array}
\right.
\]
The coordinate vector $(3x-2y,y-x)$.\\

(ii)\[
\begin{bmatrix}
    1 & 2 \\
    1 & 3
\end{bmatrix}M=\begin{bmatrix}
    4 & 6 \\
    5 & 7 \\
\end{bmatrix}
\]
\[
M=\begin{bmatrix}
    2 & 4 \\
    1 & 1
\end{bmatrix}
\]

\textbf{Exercise 3.}\\
(i)\[
T(S(\mathbf{u}_1+\mathbf{u}_2))=T(S(\mathbf{u}_1)+S(\mathbf{u}_2))=T(S(\mathbf{u}_1))+T(S(\mathbf{u}_2))
\]
\[
T(S(c\mathbf{u}))=T(cS(\mathbf{u}))=cTS(\mathbf{u})
\]
So TS is a linear transformation.\\

(ii) Assume the coordinate vector of $\mathbf{u}$ is $(c_1,\cdots , c_n)$. From definition, we can represent the coordinate vector of $S(\mathbf{u})$ with respect to $\overline{\mathbf{w}}$\[
A_S\begin{bmatrix}
    c_1\\
    \vdots\\
    c_n
\end{bmatrix}
\]
So the coordinate vector of $T(S(\mathbf{u}))$ will be \[
A_T(A_S\begin{bmatrix}
    c_1\\
    \vdots\\
    c_n
\end{bmatrix})=(A_TA_S)\begin{bmatrix}
    c_1\\
    \vdots\\
    c_n
\end{bmatrix}
\]
So $A_{TS}=A_TA_S$.\\

\textbf{Exercise 4.}\\
(i)
\begin{itemize}
    \item \[
(T+T')(\mathbf{v}_1+\mathbf{v}_2)=T(\mathbf{v}_1+\mathbf{v}_2)+T'(\mathbf{v}_1+\mathbf{v}_2)=T(\mathbf{v}_1)+T'(\mathbf{v}_1)+T(\mathbf{v}_2)+T'(\mathbf{v}_2)=(T+T')(\mathbf{v}_1)+(T+T')(\mathbf{v}_2)
\]
\[
(T+T')(c\mathbf{v})=T(c\mathbf{v})+T'(c\mathbf{v})=c(T+T')(\mathbf{v})
\]
\item \[
    cT(\mathbf{v}_1+\mathbf{v}_2)=cT(\mathbf{v}_1)+cT(\mathbf{v}_2)
\]
\[
cT(a\mathbf{v})=caT(\mathbf{v})
\]
\end{itemize}
(ii) \begin{enumerate}
    \item $T+T'(\mathbf{v})=T(\mathbf{v})+T'(\mathbf{v})=(T'+T)(\mathbf{v})$
    \item $T_1+(T_2+T_3)=T_1+T_2+T_3=(T_1+T_2)+T_3$
    \item LET $T_0(\mathbf{v}) \equiv 0$, $T_0+T = T$.
    \item For every T, we can have $T'$ that for every $\mathbf{v}$, $T(\mathbf{v})=-T'(\mathbf{v})$. So $T+T'=0$
    \item Let c=1, $1\cdot T=T$.
    \item From we have proved in (i), $c_1(c_2T)=(c_1c_2)T$.
    \item Obvious $c(T+T')=cT+cT'$.
    \item Also $(c_1+c_2)T=c_1T+c_2T$.
\end{enumerate}

(iii)The dimension is m*n.\\
Proof:Consider transforming T into the corresponding matrix $A_T$. We have proved in class that this transformation is a bijection. So each T corresponds to a unique m*n matrix $A_T$.
Since the vector space $M_{m*n}\{\mathbb{R}\}$ has dimension m*n, so do $\textbf{T(V,W)}$.\\

\end{document}