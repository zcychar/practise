\documentclass{article}

\usepackage[utf8]{inputenc}
\usepackage{amsmath}
\usepackage{amsfonts}
\usepackage{amssymb}
\usepackage{graphicx}

\title{\vspace*{-3.5cm}Homework for Linear Algebra \\October 10, 2024} 
\author{Chengyu Zhang}
\date{}

\begin{document}

\maketitle

\paragraph{Exercise 1.}
(i) $\Rightarrow$ (ii) \\
If exists a B with AB = I, that means A has n pivots.\\
So there exists a medthod of Gauss-Jordan process that produces 
\[
    D\cdots E\cdots P\cdots E\cdots A = I 
\]
Let $C=D\cdots E\cdots P\cdots E\cdots$, then we have $CA=I$.\\
(ii) $\Rightarrow$ (i) \\
Since $CA = I$ and $(CA)^T=A^TC^T$, we have $A^TC^T=I^T=I$.\\
From previous provement, we know there exists $DA^T=I$\\
So $(DA^T)^T=AD^T=I$, let $D^T=B$, AB = I exists.

\paragraph{Exercise 2.}
\[
A=\begin{bmatrix}
    1 & 1 & 0 & 0\\
    0 & 0 & 1 & 1
\end{bmatrix}
\]

\paragraph{Exercise 3.}
(i)\[
A\mathbf{x}=\mathbf{b} \Rightarrow \begin{bmatrix}
    1 & 0 & 1/5 & 7/5\\
    0 & 1 & -3/5 & 7/5\\
    0 & 0 & 0 & 0
\end{bmatrix}\mathbf{x}=\begin{bmatrix}
    4/5\\3/5\\ \lambda-5
\end{bmatrix}
\]
If $\lambda = 5$ ,it has infinite many solutions,if $\lambda \neq 5$ ,it has no solution.\\
(ii)\[
(\lambda + 3)(x_1+x_2+x_3+x_4)= (1 + \lambda + \lambda^2 + \lambda^3) \Rightarrow (x_1+x_2+x_3+x_4)=\frac{b}{c} (c \neq 0)
\]
If  $\lambda = 1$ it has infinite many solutions.\\
If $\lambda = -3$ it has no solution\\
Else we have $\begin{bmatrix}
    x_1\\x_2\\x_3\\x_4
\end{bmatrix}=\begin{bmatrix}
    \frac{\frac{b}{c}-1}{\lambda - 1}\\
    \frac{\frac{b}{c}-\lambda}{\lambda - 1}\\
    \frac{\frac{b}{c}-\lambda^2}{\lambda - 1}\\
    \frac{\frac{b}{c}-\lambda^3}{\lambda - 1}\\
\end{bmatrix}$
\paragraph{Exercise 4.}
(i) \[
P_{12}=\begin{bmatrix}
    0 & 1 & 0\\
    1 & 0 & 0\\
    0 & 0 & 1
\end{bmatrix},
E_{31}=\begin{bmatrix}
    1 & 0 & 0\\
    0 & 1 & 0\\
    -2 & 0 & 1 \\
\end{bmatrix}
E_{32}=\begin{bmatrix}
    1 & 0 & 0\\
    0 & 1 & 0\\
    0 & 1 & 1
\end{bmatrix}
\]
Failure :Temporary failure for a zero pivot.Permanent failure with no solution.(A pivot is missing and 0x = 1)\\
(ii) \[
    E_{21}=\begin{bmatrix}
        1 & 0 & 0\\
        -2 & 1 & 0\\
        0 & 0 & 1 \\
    \end{bmatrix}
    E_{31}=\begin{bmatrix}
        1 & 0 & 0\\
        0 & 1 & 0\\
        -3 & 0 & 1
    \end{bmatrix}
    E_{32}=\begin{bmatrix}
        1 & 0 & 0\\
        0 & 1 & 0\\
        0 & -2 & 1
    \end{bmatrix}
    \]
Failure :Permanent failure with infinite many solution.(A pivot is missing and 0x = 0)\\
(iii)\[
    P_{21}=\begin{bmatrix}
        0 & 1 & 0\\
        1 & 0 & 0\\
        0 & 0 & 1 \\
    \end{bmatrix}
    E_{31}=\begin{bmatrix}
        1 & 0 & 0\\
        0 & 1 & 0\\
        -6/7 & 0 & 1  
    \end{bmatrix}
    E_{32}=\begin{bmatrix}
        1 & 0 & 0\\
        0 & 1 & 0\\
        0 & 13/7 & 1
    \end{bmatrix}
\]Failure:Temporary failure for a zero pivot.

\paragraph{Exercise 5.}
(i)\[
\begin{bmatrix}
    16 & 15 & 14 & 13 & 1 & 0 & 0 & 0\\
    5 & 4 & 3 & 12 & 0 & 1 & 0 & 0\\
    6 & 1 & 2 & 11 & 0 & 0 & 1 & 0\\
    7 & 8 & 9 & 10 & 0 & 0 & 0 & 1
\end{bmatrix} \Rightarrow \begin{bmatrix}
     1 & 0 & 0 & 0&83/690&-2/15&1/6&-62/345\\
     0 & 1 & 0 & 0&34/345&11/30&-1/3&-139/690\\
     0 & 0 & 1 & 0&-17/138&-1/3&1/6&26/69\\
     0 & 0 & 0 & 1&-6/115&1/10&0&11/230\\
\end{bmatrix}
\]
(ii)
\[
    \begin{bmatrix}
        1 & 1 & 10 & 1 & 1 & 0 & 0 & 0\\
        1 & 10 & 1 & 1 & 0 & 1 & 0 & 0\\
        1 & 1 & 1 & 10 & 0 & 0 & 1 & 0\\
        10 & 1 & 1 & 1 & 0 & 0 & 0 & 1
    \end{bmatrix} \Rightarrow \begin{bmatrix}
         1 & 0 & 0 & 0&-1/117&-1/117&-1/117&4/39\\
         0 & 1 & 0 & 0&-1/117&4/39&-1/117&-1/117\\
         0 & 0 & 1 & 0&	4/39&-1/117&-1/117&-1/117\\
         0 & 0 & 0 & 1&-1/117&-1/117&4/39&-1/117\\
    \end{bmatrix}
\]
(iii)\[
    \begin{bmatrix}
        0 & 1 & 1 & 1 & 1 & 0 & 0 & 0\\
        1 & \ddots & \ddots & 1 & 0 & \ddots & \ddots & 0\\
        1 & \ddots & \ddots & 1 & 0 & \ddots & \ddots & 0\\
        1 & 1 & 1 & 0 & 0 & 0 & 0 & 1
    \end{bmatrix} \Rightarrow \begin{bmatrix}
         1 & 0 & 0 & 0&-2/3&1/3&1/3&1/3\\
         0 & \ddots & \ddots & 0&1/3& \ddots & \ddots&1/3\\
         0 & \ddots & \ddots & 0&1/3& \ddots & \ddots&1/3\\
         0 & 0 & 0 & 1&1/3&1/3&1/3&-2/3\\
    \end{bmatrix}
\]
The process are way too long to print.Only show the results.
\end{document}