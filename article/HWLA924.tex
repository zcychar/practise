\documentclass{article}

\usepackage[utf8]{inputenc}
\usepackage{amsmath}
\usepackage{amsfonts}
\usepackage{amssymb}
\usepackage{graphicx}

\title{\vspace*{-3.5cm}Homework for Linear Algebra \\September 24, 2024} 
\author{Chengyu Zhang}
\date{}

\begin{document}

\maketitle

\paragraph{Exercise 1.}
Assume $ A = 
    \begin{bmatrix}
    \mathbf{a}_1\\ \mathbf{a}_2\\ \vdots \\ \mathbf{a}_n
    \end{bmatrix} $ , $ B =
    \begin{bmatrix}
    \mathbf{b}_1 & \mathbf{b}_2 & \cdots & \mathbf{b}_m
    \end{bmatrix} $ , $ C = 
    \begin{bmatrix}
    \mathbf{c}_1\\ \mathbf{c}_2\\ \vdots \\ \mathbf{c}_n
    \end{bmatrix} $ and $ E =
    \begin{bmatrix}
    \mathbf{e}_1 & \mathbf{e}_2 & \cdots & \mathbf{e}_m
    \end{bmatrix} $\\
    \begin{enumerate}
        \item $ A\mathbf{b}_2$
        \item $ \mathbf{a}_1 B$
        \item $\mathbf{a}_3 \mathbf{b}_5$
        \item $\mathbf{c}_1 D \mathbf{e}_1$
    \end{enumerate}
\paragraph{Exercise 2.}
    \[
        (A+B)^2 =
        \begin{bmatrix}
             2 & 2 \\ 3 & 0
        \end{bmatrix}^2 = 
        \begin{bmatrix}
             10 & 4 \\ 6 & 6
        \end{bmatrix},
        A^2+2AB+B^2 =
        \begin{bmatrix}
            1 & 2 \\ 0 & 0
        \end{bmatrix} + 
        \begin{bmatrix}
            14 & 0 \\ 0 & 0
        \end{bmatrix} +
        \begin{bmatrix}
            1 & 0 \\ 3 & 0
        \end{bmatrix} =
        \begin{bmatrix}
            16 & 2 \\ 3 & 0
        \end{bmatrix}
    \]
    \[
        (A+B)(A+B)= A^2 + AB + BA + B^2
    \]
\paragraph{Exercise 3.}
    Because when we do a row swapping operation ,we actually won't change which column each entry in.\\
    When we do a  scaling or addition operation , it follows the distributive law : $ (a_1 + a_2) * c = a_1 * c + a_2 * c $. So the order is arbitrary.\\

\paragraph{Exercise 4.}
        Assume $ B = \begin{bmatrix} \mathbf{b}_1 & \mathbf{b}_2 &  \mathbf{b}_3 \end{bmatrix} $\\
        If $ AB=I$ , for $ \mathbf{b}_1= \begin{bmatrix}
            x_1\\x_2
        \end{bmatrix}$ , we have
        \[
        \left\{
             \begin{array}
                {rcl}
                2*x_1+3*x_2=1\\
                1*x_1+2*x_2=0\\
                7*x_1+100*x_2=0\\
            \end{array}
        \right.
        \]
        No solution.
\paragraph{Exercise 5.}
    For $ \mathbf{x}= \begin{bmatrix}
        1 \\ 0 \\ 0 \\\vdots
    \end{bmatrix}$ , $ A\mathbf{x} $ equals the first column of A. So if for all the x with one entry 1 and the rest 0 , we have $ A\mathbf{x} = B\mathbf{x}$ , we have A = B.
\end{document}