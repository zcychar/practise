\documentclass{article}

\usepackage[utf8]{inputenc}
\usepackage{amsmath}
\usepackage{amsfonts}
\usepackage{amssymb}
\usepackage{graphicx}

\title{\vspace*{-3.5cm}Homework for Linear Algebra \\October 24, 2024} 
\author{Chengyu Zhang}
\date{}

\begin{document}

\maketitle

\paragraph{Exercise 1.}
    \paragraph{1.1}
        \textit{
            \[
            \mathbf{v}\mathbf{w}=\mathbf{v}^T\mathbf{w}
            \]
            See the result as a 1*1 matrix.
            \[
            \mathbf{v}^T\mathbf{w}=(\mathbf{v}^T\mathbf{w})^T=\mathbf{w}^T\mathbf{v}=\mathbf{w}\mathbf{v}
            \]
        }
    \paragraph{1.2}
        \[
        (\mathbf{u}+\mathbf{v})\mathbf{w}=\sum_{i=1}^{m}(u_i+v_i)w_i=\sum_{i=1}^{m}u_iw_i+\sum_{i=1}^{m}v_iw_i=\mathbf{u}\mathbf{w}+\mathbf{v}\mathbf{w} 
        \]
    \paragraph{1.3}
        \[
        c\mathbf{u}\cdot\mathbf{v}=\sum_{i=1}^{m}(cu_i)v_i=\sum_{i=1}^{m}c(u_iv_i)=c(\mathbf{u}\mathbf{v})
        \]
    \paragraph{1.4}
        \[
        \mathbf{u}\cdot \mathbf{u}=\sum_{i=1}^{m} u_i^2 \geq 0
        \]
    When all $u_i=0$, which means $\mathbf{u}=\mathbf{0}$, $\mathbf{u}\cdot \mathbf{u}=0$
\paragraph{Exercise 2.}
    \[
    \Vert \mathbf{u}+\mathbf{v} \Vert^2 = ( \mathbf{u}+\mathbf{v} )^2 =  \Vert \mathbf{u} \Vert^2 +  \Vert \mathbf{v} \Vert^2 + 2\cdot \mathbf{u}\cdot\mathbf{v}
    \]
    \[
    \Rightarrow 
    \]
    \[
    \mathbf{u}\cdot\mathbf{v}=0 \Rightarrow \Vert \mathbf{u}+\mathbf{v} \Vert^2 =  \Vert \mathbf{u} \Vert^2 +  \Vert \mathbf{v} \Vert^2
    \]
    \[
    \Leftarrow
    \]
    \[
        \Vert \mathbf{u}+\mathbf{v} \Vert^2 =  \Vert \mathbf{u} \Vert^2 +  \Vert \mathbf{v} \Vert^2 \Rightarrow  \mathbf{u}\cdot\mathbf{v}=0 \Rightarrow \mathbf{u} \perp \mathbf{v}
    \]
\paragraph{Exercise 3.}
    \paragraph{3.1}
        \textit{
            Since $rank(A)=n$ and A has n columns, we can know that the column vectors of A are linearly independent.\\
            So $dim(C(A))=dim(V)=n$, and there will be a matrix A whose column vectors comsist a basis of V.
        }
    \paragraph{3.2}
        \textit{
            Name a basis of V $\{\mathbf{a}_1 \cdots \mathbf{a}_n\}$.We need to find 
            \[
            \mathbf{p}=\hat{x}_1\mathbf{a}_1+\cdots +\hat{x}_n\mathbf{a}_n
            \]
            Let
            \[
            \mathbf{\hat{x}}=(\hat{x}_1,\cdots \hat{x}_n) , A=\begin{bmatrix}
                \mathbf{a}_1 & \cdots \mathbf{a}_n
            \end{bmatrix}
            \]
            Then $\mathbf{p}=A\mathbf{\hat{x}}$
            \[
            \mathbf{e} \perp V \Rightarrow \mathbf{v} \perp \{\mathbf{a}_1\cdots\mathbf{a}_n\} \Rightarrow \begin{bmatrix}
                \mathbf{a}^T_1\\
                \vdots\\
                \mathbf{a}^T_n
            \end{bmatrix}\cdot \mathbf{e} = \mathbf{0}
            \]
            \[
            A^T \mathbf{e} = A^T (\mathbf{v}-A\mathbf{\hat{x}}) = \mathbf{0} \Rightarrow A^T \mathbf{v}= A^TA\mathbf{\hat{x}} \Rightarrow (A^TA)^{-1}A^T\mathbf{v}\mathbf{\hat{x}}
            \]
            So there exists a unique way to get the $\mathbf{\hat{x}}$, and the $\mathbf{e}$ is unique.
        }
    \paragraph{3.3}
        \[
        \Vert \mathbf{v}-\mathbf{u} \Vert = \Vert \mathbf{v}-\mathbf{p}+\mathbf{p}-\mathbf{u} \Vert
        \]
        \[
        \Vert \mathbf{v}-\mathbf{p}+\mathbf{p}-\mathbf{u} \Vert^2 = (\mathbf{v}-\mathbf{p})^2 +(\mathbf{p}-\mathbf{w})^2 + 2*\mathbf{e}(\mathbf{p}-\mathbf{u})
        \]
        Since 
        \[
        \mathbf{p},\mathbf{u} \in V \mathbf{e} \perp V 
        \]
        \[
        \Rightarrow \mathbf{e}\perp \mathbf{p} , \mathbf{e} \perp \mathbf{u}
        \]
        So
        \[
            \Vert \mathbf{v}-\mathbf{u} \Vert^2  - \Vert\mathbf{e}\Vert^2  = (\mathbf{p}-\mathbf{w})^2 \geq 0 
        \]
        When $\mathbf{p}=\mathbf{w}$, we have $\Vert \mathbf{v}-\mathbf{u} \Vert = \Vert (\mathbf{e}) \Vert$
    \paragraph{3.4}
        \textit{
            From the definition the dist is the shortest vector prependicular to V,and the $\mathbf{e}$ in 3.2 satisfies.So 
            \[
            dist(\mathbf{v},V)=min \Vert \mathbf{v}-\mathbf{u} \Vert = \Vert \mathbf{e} \Vert
            \]
        }
\paragraph{Exercise 4.}
    \textit{
        Using
        \[
        \mathbf{p}=\frac{\Vert \mathbf{b} \Vert cos \theta }{\Vert \mathbf{a} \Vert} \mathbf{a}
        \]
        \[
        \mathbf{p}_1=\begin{bmatrix}
            \frac{5}{3} \\
            \frac{5}{3} \\
            \frac{5}{3} \\
        \end{bmatrix} , 
        \mathbf{e}_1=\begin{bmatrix}
            -\frac{2}{3} \\
            \frac{1}{3} \\
            \frac{1}{3} \\
        \end{bmatrix}
        \]
        \[
            \mathbf{p}_2=\begin{bmatrix}
                1\\
                3\\
                1\\
            \end{bmatrix} , 
            \mathbf{e}_2=\begin{bmatrix}
                0\\0\\0
            \end{bmatrix}
            \]
    }
\paragraph{Exercise 5.}
    \textit{
        Using
        \[
        \mathbf{p}=A(A^TA)^{-1}A^T\mathbf{b}
        \]
        \[
        (A^TA)^{-1}=\begin{bmatrix}
           2 & -1\\-1 & 1
        \end{bmatrix},
        A(A^TA)^{-1}A^T= \begin{bmatrix}
            1 & 0 & 0\\
            0 & 1 & 0 \\
            0 & 0 & 0
         \end{bmatrix}
        \mathbf{p_1}=\begin{bmatrix}
            2\\3\\0
        \end{bmatrix}
        \]
        \[
            (A^TA)^{-1}=\begin{bmatrix}
                3/2 & -1\\ -1 & 1
             \end{bmatrix},
             A(A^TA)^{-1}A^T= \begin{bmatrix}
                1/2 & 1/2 & 0\\
                1/2 & 1/2 & 0 \\
                0 & 0 & 1
             \end{bmatrix}
             \mathbf{p_2}=\begin{bmatrix}
                 4\\4\\6
             \end{bmatrix}
        \]
    }
\end{document}