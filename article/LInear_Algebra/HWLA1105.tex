\documentclass{article}

\usepackage[utf8]{inputenc}
\usepackage{amsmath}
\usepackage{amsfonts}
\usepackage{amssymb}
\usepackage{graphicx}

\title{\vspace*{-3.5cm}Homework for Linear Algebra \\November 5, 2024} 
\author{Chengyu Zhang}
\date{}

\begin{document}

\maketitle
\textbf{Exercise 1.}\\

    We have known the equation $det(A)=det(A^T)$. And
    \[
    det(-A)=\sum_{\sigma \in Perm(n)}^{}(-1)^{\tau(\sigma)}(-a_{\sigma(1)1})\cdots(-a_{\sigma(n)n})=\sum_{\sigma \in Perm(n)}^{}(-1)^{\tau(\sigma)}(-1)^n(a_{\sigma(1)1})\cdots(a_{\sigma(n)n})
    \]

    If n is odd, $det(-A)=-det(A)=det(A)$. So $det(A)=0$. If  n is even, $det(-A)=det(A)$, not true.

\textbf{Exercise 2.}\\

    Do elementary row operation to the matrix
    \[
    \begin{bmatrix}
        1 & a & a^2 \\
        1 & b & b^2 \\
        1 & c & c^2 \\
    \end{bmatrix}
    \Rightarrow
    \begin{bmatrix}
        1 & a & a^2 \\ 
        0 & b-a & b^2-a^2\\
        0 & c-a & c^2-a^2
    \end{bmatrix}
    \Rightarrow
    \begin{bmatrix}
        1 & a & a^2 \\ 
        0 & b-a & b^2-a^2\\
        0 & 0 & c^2-a^2-\frac{c-a}{b-a}(b^2-a^2)
    \end{bmatrix}
    =
    \begin{bmatrix}
        1 & a & a^2 \\ 
        0 & b-a & b^2-a^2\\
        0 & 0 & (c-a)(c-b)
    \end{bmatrix}
    \]
    \[
    det(A)=(b-a)(c-a)(c-b)
    \]

\textbf{Exercise 3.}\\

    Calculate the matrix's determinant by big formula
    \[
    det(A)=\sum_{\sigma \in Perm(n)}^{}(-1)^{\tau(\sigma)}a_{\sigma(1)1}\cdots a_{\sigma(n)n}=\sum_{\sigma \in Perm(n)}^{}(-1)^{\tau(\sigma)}
    \]
    Since A has two same columns, det(A)=0, there's det(A)=0. Since 
    \[
    (-1)^{\tau(\sigma)}=
    \left\{
    \begin{array}{l}
     -1 (\sigma \text{ is odd }) \\
    1 (\sigma \text{ is even })       
    \end{array}
    \right.
    \]
    So there're equal number of odd and even permuntations to maintain the det=0.\\

\textbf{Exercise 4.}\\

    First prove for any column vector of A and row column of A*
    \[
    \mathbf{a}_i * (\mathbf{a}^*)^T_j\left\{
        \begin{aligned}
        0 &(i\neq j) \\
        det(A) &(i = j)       
        \end{aligned}
        \right.
    \]
    
    When $i=j$, it is the form of cofactor formula.\\

    When $i\neq j $, it equals the cofactor formula of a matrix that copies the jth column to ith column. And from definition, we know its determinant equals 0.\\


\textbf{Exercise 5.}\\

Use the equation proved in Exercise 4. Apply cofactor formula to calulate.
\[
A^{-1}=\frac{A^*}{det(A)}
\]
\begin{enumerate}
    \item \[
    A^*=\begin{bmatrix}
        4 & -2 \\
        -3 & 1
    \end{bmatrix} , det(A) = 1*(-2)=-2 , A^{-1}=\begin{bmatrix}
        -2 & 1\\
        \frac{3}{2} & -\frac{1}{2}
    \end{bmatrix}
    \]
    \item \[
        A^*=\begin{bmatrix}
            -2 & 17 \\
            -3 & 19
        \end{bmatrix}
        ,
        det(A)=19*\frac{13}{19}=13 , 
        A^{-1}=\begin{bmatrix}
            -\frac{2}{13} & \frac{17}{13}\\
            -\frac{3}{13} & \frac{19}{13}
        \end{bmatrix}
    \]
    \item \[
        A^*=\begin{bmatrix}
            5 & -3 \\
            0 & 1
        \end{bmatrix}
        ,
        det(A)=1*5=5 , 
        A^{-1}=\begin{bmatrix}
            1 & 0 \\
            -\frac{3}{5}& \frac{1}{5}
        \end{bmatrix}
    \]
    \item \[
        A^*=\begin{bmatrix}
            -1 & -1 & 2 \\
            -2 & 1 & -2 \\                                               
            2 & -1 & -1
        \end{bmatrix}
        ,
        det(A)=1*(-1)*3=-3, 
        A^{-1}=\begin{bmatrix}
            \frac{1}{3} & \frac{1}{3} & -\frac{2}{3}\\
            \frac{2}{3} & -\frac{1}{3} & \frac{2}{3}\\
            -\frac{2}{3} & \frac{1}{3} & \frac{1}{3}
        \end{bmatrix}
    \]
\end{enumerate}
\end{document}