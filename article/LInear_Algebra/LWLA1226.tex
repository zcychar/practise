\documentclass{article}

\usepackage[utf8]{inputenc}
\usepackage{amsmath}
\usepackage{amsfonts}
\usepackage{amssymb}
\usepackage{graphicx}

\title{\vspace*{-3.5cm}Homework for Linear Algebra \\December 26, 2024} 
\author{Chengyu Zhang}
\date{}

\begin{document}

\maketitle
\textbf{Exercise 1.}\\
  \textbf{1.1} For any u,since the second vector is zero vector\[
  \langle\mathbf{u},\mathbf{0}\rangle=\langle\mathbf{u},0\mathbf{0}\rangle=0\langle\mathbf{u},\mathbf{0}\rangle=0
  \]
  \textbf{1.2} If $\mathbf{v}\neq 0$, for any nonzero v, we have 
  \[
  \cos \theta = \frac{\langle\mathbf{u},\mathbf{v}\rangle}{||\mathbf{u}||||\mathbf{v}||} = 0
  \]
  But since the choose of $\mathbf{u}$ is arbitrary, we can always find $\mathbf{u}$ that the cos of angle of u and v is not zero. Contradict. So v must be zero. \\
  \textbf{1.3} \[
    \langle\mathbf{u},\mathbf{w}\rangle=\langle\mathbf{v},\mathbf{w}\rangle Rightarrow \langle\mathbf{u}-\mathbf{v},\mathbf{w}\rangle=0
  \]
  From what we have known in 1.2, $\mathbf{u}-\mathbf{v}=0$ so $\mathbf{u}=\mathbf{v}$\\

\textbf{Exercise 2}\\

  \textbf{2.1} Since\[
    \cos \theta = \frac{\langle\mathbf{u},\mathbf{v}\rangle}{||\mathbf{u}||||\mathbf{v}||} 
  \]
  we have
  \[
    \cos^2 \theta ||\mathbf{u}|| ^2||\mathbf{v}|| ^2= \langle\mathbf{u},\mathbf{v}\rangle^2
  \]
  And from what we known\[
    \langle\mathbf{u},\mathbf{u}\rangle=||\mathbf{u}|| ^2 ,  \langle\mathbf{v},\mathbf{v}\rangle=||\mathbf{v}|| ^2
  \]
  So when the equation \[
    \langle\mathbf{u},\mathbf{v}\rangle^2=||\mathbf{u}|| ^2||\mathbf{v}|| ^2
  \]
  holds, we have $\cos^2 \theta =1$, which equivlant to u and v are linear dependent.\\

  \textbf{2.2} 
  \begin{itemize}
    \item \[
        ||\mathbf{u}+\mathbf{v}||=\langle\mathbf{u}+\mathbf{v},\mathbf{u}+\mathbf{v}\rangle=\langle\mathbf{u},\mathbf{u}\rangle+2\langle\mathbf{u},\mathbf{v}\rangle+\langle\mathbf{v},\mathbf{v}\rangle
    \]
    Since $\langle\mathbf{u},\mathbf{v}\rangle=0$, also $\langle\mathbf{u},\mathbf{u}\rangle=||\mathbf{u}||^2$, we have $||\mathbf{u}+\mathbf{v}||=||\mathbf{u}||^2+||\mathbf{v}||^2$
    \item From the same equation above, we have $\langle\mathbf{u},\mathbf{v}\rangle=0$,so we know $\mathbf{u}\perp\mathbf{v}$.
  \end{itemize}

\textbf{Exercise 3.}\\

\begin{enumerate}
    \item Since the fraction is curculant symmetric, when we change the position iof $\mathbf{v}$ and $\mathbf{u}$, the result wont change.
    \item Assume $\mathbf{x}'=[x'_1,x'_2]^T$
    \[
    \langle\mathbf{x}+\mathbf{x}',\mathbf{y}\rangle=(x_1+x'_1)y_1-(x_2+x'_2)y_1-(x_1+x'_1)y_2+3(x_2+x'_2)y_2
    \]
    \[
    =(x_1y_1-x_2y_1-x_1y_2+3x_2y_2)+(x'_1y_1-x'_2y_1-x'_1y_2+3x'_2y_2)= \langle\mathbf{x},\mathbf{y}\rangle+ \langle\mathbf{x}',\mathbf{y}\rangle
    \]
    \item \[
        \langle c\mathbf{x},\mathbf{y}\rangle=(cx_1y_1-cx_2y_1-cx_1y_2+3cx_2y_2)=c(x_1y_1-x_2y_1-x_1y_2+3x_2y_2)=c\langle \mathbf{x},\mathbf{y}\rangle
    \]
    \item \[
        \langle \mathbf{x},\mathbf{x}\rangle=x_1^2-x_1x_2-x_1x_2+3x_2^2=(x_1-x_2)^2+2x_2^2 \geq 0
    \]
    And only when $x_1=x_2=0$, which x=0 , the fraction equals zero.
\end{enumerate}
So the function sufficients a innerproduct.\\

\textbf{Exercise 4.}\\

\begin{enumerate}
    \item From the form of equation, we have $\langle \mathbf{x},\mathbf{y}\rangle=\langle \mathbf{y},\mathbf{x}\rangle$
    \item Assume a function $f'[0,1]\rightarrow \mathbb{R}$ Use the property of intergral we have
    \[
        \langle f+f',g\rangle=\int_{0}^{1}(f(x)+f'(x))g(x)\,dx=\int_{0}^{1}f(x)g(x)\,dx+\int_{0}^{1}f'(x)g(x)\,dx=\langle f,g\rangle+\langle f',g\rangle
    \]
    \item \[
        \langle cf,g\rangle=\int_{0}^{1}cf(x)g(x)\,dx=c\int_{0}^{1}f(x)g(x)\,dx=c\langle f,g\rangle
    \]
    \item \[
    \langle f,f \rangle=\int_{0}^{1}f(x)^2\,dx \geq 0
    \]
    If and only if $f\equiv 0$, the fraction equals zero.
\end{enumerate}
So it is an innerproduct.\\

For the equation \[
c_1f_1+c_2f_2+c_3f_3=0 \Rightarrow c_1+c_2(x-1)+c_3(x-1)^2=0
\]
Since the choose of x is arbitrary, the only way to hold the equation is $c_1=c_2=c_3=0$, so the three functions are linearly independent.\\

From what we have learnt $dim(span\{f_1,f_2,f_3\})=3$ since $||f_1||=\sqrt{\langle f_1,f_1\rangle}=1$, we choose it as the first unit$g_1$.
Using generalized Gram-Schmidt, 
\[
g_2=f_2-\frac{\langle g_1,f_2\rangle}{\langle g_1,g_1\rangle}g_1=x-\frac{1}{2}
\]
\[
g_3=f_3-\frac{\langle g_1,f_3\rangle}{\langle g_1,g_1\rangle}g_1-\frac{\langle g_2,f_3\rangle}{\langle g_2,g_2\rangle}g_2=x^2-x+\frac{5}{6}
\]
orthornormalize them, then $g_1=1,g_2=2\sqrt{3}(x-\frac{1}{2}),g_3=\frac{2\sqrt{5}}{3}(x^2-x+\frac{5}{6})$
And $g_1,g_2,g_3$ consists a orthornormal basis of $span\{f_1,f_2,f_3\}$.
\end{document}