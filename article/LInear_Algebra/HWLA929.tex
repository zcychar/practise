\documentclass{article}

\usepackage[utf8]{inputenc}
\usepackage{amsmath}
\usepackage{amsfonts}
\usepackage{amssymb}
\usepackage{graphicx}

\title{\vspace*{-3.5cm}Homework for Linear Algebra \\September 29, 2024} 
\author{Chengyu Zhang}
\date{}

\begin{document}

\maketitle

\paragraph{Exercise 1.}Total m*n matrices.

  \[
          \begin{bmatrix}
            1 & 0 &\cdots & 0\\
            0 & 0 &\cdots & 0\\
            \vdots & \vdots & \ddots & \vdots\\
            0 & 0 &\cdots & 0
          \end{bmatrix},
          \begin{bmatrix}
            0 & 1 &\cdots & 0\\
            0 & 0 &\cdots & 0\\
            \vdots & \vdots & \ddots & \vdots\\
            0 & 0 &\cdots & 0
          \end{bmatrix}
          \cdots 
          \underbrace{
          \begin{bmatrix}
            0 & 0 &\cdots & 0\\
            0 & 0 &\cdots & 0\\
            \vdots & \vdots & \ddots & \vdots\\
            0 & 0 &\cdots &1
          \end{bmatrix}
        }_{\text{n}}
        \left.
        \begin{array}{c}
        \\
        \\
        \\
        \\
        \\
        \end{array}
        \right\}{\text{m}}
  \]

  \paragraph{Exercise 2.} The given vector space has one basis element. Let it be $\mathbf{e}$\\
  So for any $x\in V$, we can write $x$ as a "scalar multiple" of $\mathbf{e}$. That is, there exists a scalar $\alpha$ such that $x = \alpha\otimes\mathbf{e}=\mathbf{e}^\alpha$.

  
  \paragraph{Exercise 3.} Yes.
  Define a vector space \[\mathbf{V} = \{0,1\} \] with  \[x \oplus x'=(x+x')mod 2, x \otimes x'=(x*x') mod 2.\] \\
  It only has two vectors{(0),(1)}.

  \paragraph{Exercise 4.} 
  $(\mathbf{v}_1,\mathbf{v}_2,\mathbf{v}_4),(\mathbf{v}_1,\mathbf{v}_3,\mathbf{v}_5),(\mathbf{v}_2,\mathbf{v}_3,\mathbf{v}_6),(\mathbf{v}_4,\mathbf{v}_5,\mathbf{v}_6)$ ,those groups of vectors are each linear dependent.So the largest possible number is 3.
  \paragraph{Exercise 5.}
  (i)\[
    \left\{
    \begin{array}{c}
    2x_1+3x_2+4x_3+5x_4+1x_5=0\\
    x_2+x_4=0\\
    \end{array}\right.
    \rightarrow
    \left\{
    \begin{array}{c}
    2x_1+3x_3+2x_4+1x_5=0\\
    x_2+x_4=0\\
    \end{array}\right.
  \]
  \[
  \mathbf{N}(A)=span
  \left\{
    \begin{bmatrix}
    -2\\0\\1\\0\\0
    \end{bmatrix}
    \begin{bmatrix}
      -2\\-1\\0\\2\\0
      \end{bmatrix}   
      \begin{bmatrix}
        -1\\0\\0\\0\\2
        \end{bmatrix} 
    \right\},
    \mathbf{C}(A)=span
    \left\{
      \begin{bmatrix}
      1\\0
      \end{bmatrix}
      \begin{bmatrix}
      3\\1
        \end{bmatrix}   
      \right\}
  \]
  (ii)\[
    \mathbf{N}(A)=span
    \left\{
      \begin{bmatrix}
      1\\0\\0\\2
      \end{bmatrix}
      \begin{bmatrix}
        0\\0\\2\\1
        \end{bmatrix}   
        \begin{bmatrix}
          0\\2\\0\\-1
          \end{bmatrix} 
      \right\},
      \mathbf{C}(A)=span
      \left\{
        \begin{bmatrix}
        1\\0\\0\\0\\0
        \end{bmatrix}
        \begin{bmatrix}
        0\\0\\1\\1\\0
          \end{bmatrix}   
          \begin{bmatrix}
        0\\0\\0\\2\\1
            \end{bmatrix} 
          \begin{bmatrix}
              0\\1\\0\\0\\0
        \end{bmatrix} 
        \right\}
  \]
  The null spaces and column spaces are written as a span of their basis. So the vectors above are their basis.
  \paragraph{Exercise 6.}

  \[
    \left\{
      \begin{align}
      x_1+(1-k)x_2+(1-k)x_3+(1-k)x_4&=0\\
      x_2+(1-k)x_3+(1-k)x_4&=0\\
      x_3+(1-k)x_4&=0\\
      -kx_1-kx_2-kx_3+(1-k)x_4&=0\\
      \end{align}\right.
      \rightarrow
      \left\{
      \begin{align}
      x_1-kx_2&=0\\
      x_2-kx_3&=0\\
      x_3+(1-k)x_4&=0\\
      -kx_1-kx_2-kx_3+(1-k)x_4&=0
      \end{align}\right.
\]
\[
\rightarrow
(k^3+k^2+k-1)(1-k)x_4=0
\]

If $k=1$ ,the equation system will have infinite many solutions.So the vectors are linear dependent.\\

When $k \neq 1$ the vectors are linear independent.
\paragraph{Exercise 7.}\\

(i)Assume $ \mathbf{v} \in span(S_1)$,$\mathbf{v}$ can be expressed by linear combination of vectors in span(S1). Since $S_1 \subseteq S_2$, $ \mathbf{v} \in S_2$ .So $\mathbf{v}$ can be written in a linear combination form of vectors in $span(S_2)$, which is just how $span(S_2)$ is defined.\\
So every vector in $span(S_1)$ is in $span(S_2)$. $ span(S_1) \subseteq span(S_2) $.\\

(ii)If $\text{span}(S_1) \neq \text{span}(\text{span}(S_2))$, it means that $\text{span}(S_1)$ isn't the \mathbf{SMALLEST} subspace of $V$ containing $S_1$. Contradict!\\

(iii)If $S_1 \subseteq \text{span}(S_2)$ , so $\text{span}(S_1) \subseteq S_1 \subseteq \text{span}(S_2)$

\paragraph{Exercise 8.}\\

  (i)$S_1$ can be represented by $span(S_1)$.If $ span(S_1) \subseteq span(S_2)$.\\Every $\mathbf{v}\in S_1$ can be represented by $span(S_2)$.So $S_1$ can be represented by $S_2$, if and only if  $ span(S_1) \subseteq span(S_2)$.\\

  (ii)If $S_1$ can be represented by $S_2$ ,then all the $v\in span(S_1)\subseteq S_1$ can also be represented by $span(S_2)$.Because all $\mathbf{v}\in S$ can be represented by $span(S_1)$ , so it can be represented by $span(S_2)$ \\
  So S can be represented by S_2.\\

  (iii) If $\mathbf{v}$ can be represented by $S \setminus \mathbf{v}$ ,then $ \mathbf{v} \notin span(S)$. Because any vector in span(S) cannot be represented by other vectors in span(S).So we have \[span(S \setminus \mathbf{v}) = span(S)\]

\end{document}