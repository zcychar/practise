\documentclass{article}

\usepackage[utf8]{inputenc}
\usepackage{amsmath}
\usepackage{amsfonts}
\usepackage{amssymb}
\usepackage{graphicx}

\title{\vspace*{-3.5cm}Homework for Linear Algebra \\October 8, 2024} 
\author{Chengyu Zhang}
\date{}

\begin{document}

\maketitle

\paragraph{Exercise 1.}
(i)
\begin{itemize}
\item$S'$ is a basis for $\text{span}(S) \Rightarrow S'$ is maximally linearly independent:\\
From the definition of basis, $S'$ is linearly independent.$S'$ satisfies the definition of maximally linearly independent.
For any $\mathbf{v} \in span(S)\backslash S'$, it can be represented by a linear combination of vectors in $S'$ ,so $S' \cup \mathbf{v}$ is linear dependent. 
\item $S'$ is maximally linearly independent $\Rightarrow S'$ is a basis for $\text{span}(S)$: \\
From the definition of maximally linearly independent, for any $\mathbf{v} \in S\backslash S'$ ,it can be repensented by a linear combination of vectors in $S'$.\\
For any vector in $span(S)$,it can be repensented by vectors in $S$ ,and inductively, be repensented by vectors in $S'$. Also $S'$ is linear independent, so it satisfies a basis for $span(S)$.
\end{itemize}

(ii)
\begin{itemize}
    \item$S$ is a basis for $V \Rightarrow S$ is maximally linearly independent in $V$:\\
    Since $S$ is a basis, it is linear independent and spans $V$.\\
    So if it is not maximally linear independent, there exists $\mathbf{v} \in V \setminus S$ that $S\cup \mathbf{v}$ is linear independent.\\
    This contradicts that $S$ spans $V$.\\
    \item $S$ is maximally linearly independent $\Rightarrow S$ is a basis for $V$: \\
    If $S$ isn't a basis for $V$, there exists $ \mathbf{v} \in V $that $\mathbf{v} \notin span(S)$.\\
    So $ S\cup \mathbf{v}$ remains linear independent.Contradict the definition.
    \end{itemize}
(iii)Assume $S'$ is a subset of $S$ and is always linear independent. To build a $S'$, we inductively add vectors in $S$ one by one.Until no more vectors can be added.Since $S$ is a finite set ,the process will always have a end.\\
So for any $\mathbf{v} \in S \setminus S'$, $S'\cup \mathbf{v}$ will lead to linear dependent.\\
Now $S'$ satisfies the definition of maximally linear independent.

\paragraph{Exercise 2.}
First,consider the situation that any of $\mathbf{v},\mathbf{w}=\mathbf{0}$.\\Obviously, (i) or (ii) will hold in this situation.\\
Under the condition that none of $\mathbf{v},\mathbf{w}=\mathbf{0}$.Since $\mathbf{u}_1\cdots\mathbf{u}_n,\mathbf{v},\mathbf{w}$ are linearly dependent, there exists 
\[
    c_1\mathbf{u}_1+\cdots+c_n\mathbf{u}_n+a\mathbf{v}+b\mathbf{w}=\mathbf{0} 
\]
(Assume not all $c1 \cdots cn,a,b$ are zero.)\\
Since $\mathbf{u}_1\cdots\mathbf{u}_n$ are linearly dependent ,the only way that
\[
    c_1\mathbf{u}_1+\cdots+c_n\mathbf{u}_n=\mathbf{0} 
\]
is all the coefficients are zero. So a,b cannot all be zero. So we have:\\
If $b=0,a\neq 0$, $c_1\mathbf{u}_1+\cdots+c_n\mathbf{u}_n=-a\mathbf{v}$.(i) holds.\\
If $a=0,b\neq 0$, $c_1\mathbf{u}_1+\cdots+c_n\mathbf{u}_n=-b\mathbf{w}$.(ii) holds.\\
If $a\neq 0,b\neq 0$, $c_1\mathbf{u}_1+\cdots+c_n\mathbf{u}_n=-b\mathbf{w}-a\mathbf{v}$.(iii) holds.

\paragraph{Exercise 3.}$\mathbf{v}_3$
\paragraph{Exercise 4.}Only if $V=Z$,or the vectors consist the basis can be transformed by linear combination.
\paragraph{Exercise 5.}(1,1,0,1),(10,7,2,3),(0,0,1,0),(0,0,0,1)
\paragraph{Exercise 6.}\begin{itemize}
\item $\mathbf{W} \subsetneq \mathbf{V} \Rightarrow dim(\mathbf{W}) < dim(\mathbf{V})$\\
    Assume $dim(\mathbf{W}) = dim(\mathbf{V})$.\\ Since $\mathbf{W} \subsetneq \mathbf{V}$, there exists $\mathbf{v} \in \mathbf{V} \setminus \mathbf{W}$ and cannot be represented by the basis of $\mathbf{W}$.So $dim(\mathbf{W}\cup \mathbf{v})=dim(\mathbf{W})+1>\mathbf{V}$\\
    Since $\mathbf{v} \cup \mathbf{W} \in \mathbf{V}$, $dim(\mathbf{v} \cup \mathbf{W} ) \leq dim (\mathbf{V})$. Contradict!
\item $\mathbf{W} \subsetneq \mathbf{V} \Leftarrow dim(\mathbf{W}) < dim(\mathbf{V})$\\
    Assume $\mathbf{W}=\mathbf{V}$, from the definition of basis ,if $dim(\mathbf{W}) < dim(\mathbf{V})$, then the current basis of V doesn't hold.Contradict!
\end{itemize} 

\paragraph{Exercise 7.}
(i)Since all the column vectors of A $\in R^m$ , so \[
    dim(C(A))<=dim(R^m)=m.
\]
So the basis of $C(A)$ has at most m vectors. Since we have n column vectors, they are linearly dependent.\\
(ii)For A $\begin{bmatrix}
    \mathbf{a}_1 & \cdots & \mathbf{a}_n 
\end{bmatrix}$, we have the linear combination:
\[
x_1\mathbf{a}_1+\cdots + x_n\mathbf{a}_n = \mathbf{b}
\]
And since A is linearly dependent ,there exists nonzero solution to :
\[
    c_1\mathbf{a}_1+\cdots + c_n\mathbf{a}_n = \mathbf{0}
\]
So if $A\mathbf{x}=\mathbf{b}$ holds, $A \begin{bmatrix}
    x_1+k*c_1\\
    \vdots\\
    x_n+k*c_n
\end{bmatrix} = \mathbf{b}$ $(k\in Z)$ holds too ,it has infinite many solutions.\\
If $dim(C(A))<m$ ,which means $span(C(A)) \subsetneq R^m$,there exists $\mathbf{v}\in R^m \setminus span(C(A))$ that has no solution.\\
(iii)\begin{itemize}
    \item column rank of A is m $\Rightarrow$ $A\mathbf{x}=\mathbf{b}$ has infinite many solutions\\
        From previous provements,we know that if $dim(C(A))=m=R^m$, $span(C(A))=R^m$, and if the equation has a solution, it will have infinite many solutions.
    \item column rank of A is m $\Leftarrow$ $A\mathbf{x}=\mathbf{b}$ has infinite many solutions\\
        If there always exists a solution, that means $R^m \subseteq span(C(A))$. Since the column vectors $\in R^m$ So $span(C(A)) = R^m $.So the column rank of A is m.
\end{itemize}

\paragraph{Exercise 8.}
    Write \[
    A=\begin{bmatrix}
        \mathbf{a}_1\\
        \vdots\\
        \mathbf{a}_m
    \end{bmatrix}
    \]As row vectors in $R^n$.So does \[
    I=\begin{bmatrix}
    \mathbf{e}_1\\
    \vdots\\
    \mathbf{e}_n
    \end{bmatrix}
    \].
    Assume there exist B thats holds the equation, then
    \[
    \mathbf{e}_1 \cdots \mathbf{e}_n \in span(\{\mathbf{a}_1 \cdots \mathbf{a}_m\})
    \]
    So it implies
    \[
    R^n=span(\{ \mathbf{e}_1\cdots \mathbf{e}_n\}) \subseteq span(span(\{\mathbf{a}_1 \cdots \mathbf{a}_m\}))=span(\{\mathbf{a}_1 \cdots \mathbf{a}_m\}) \subseteq R^n
    \]
    So the row rank of A is n.\\

    If the row rank of A is less than n.That means $span(\{\mathbf{a}_1 \cdots \mathbf{a}_m\}) \subsetneq R^n$. So there must exist a $\mathbf{e}_i \in R^n \setminus R(A)$ that can not be reresented by linear combinations of the row vectors.
    Unless $dim(span(\{\mathbf{a}_1 \cdots \mathbf{a}_m\}))=dim(R^n)$.So the row rank must be n.
\end{document}